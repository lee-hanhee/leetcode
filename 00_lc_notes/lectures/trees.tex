\subsection{Binary Search Tree (BST)}
\begin{summary}
    \begin{itemize}
        \item A binary tree where for each node, left subtree values are smaller, and right subtree values are larger.
        \item \textbf{Balanced vs. Unbalanced:} 
            \begin{equation*}
                O(\log (n)) \; \text{(balanced)} \leq O(h) \leq O(n) \; \text{(unbalanced)}
            \end{equation*}
    \end{itemize}
\end{summary}

\begin{algo}
\begin{lstlisting}
class Node:
    def __init__(self, key):
        self.val = key
        self.left = None
        self.right = None

class BST:
    def __init__(self):
        self.root = None

    def operations(self,_):
        pass
\end{lstlisting}
\end{algo}
\newpage

\subsection{Operations}
\begin{summary}
    \begin{center}
        \begin{tabular}{ll}
            \toprule
            \textbf{Operation} & \textbf{Time Complexity} \\
            \midrule
            Search & $O(h)$ \\
            \midrule
            Insert & $O(h)$ \\
            \midrule
            Delete & $O(h)$ \\
            \midrule
            Find Min/Max & $O(h)$ \\
            \midrule
            In-order Traversal & $O(n)$ \\
            \midrule
            Pre-order Traversal & $O(n)$ \\
            \midrule
            Post-order Traversal & $O(n)$ \\
            \midrule
            Level-order Traversal & $O(n)$ \\
            \bottomrule
        \end{tabular}
    \end{center}
\end{summary}

\subsubsection{Search}
\begin{algo}
\begin{lstlisting}
def search(self, key):
    current = self.root
    while current:
        if key == current.val:
            return current
        elif key < current.val:
            current = current.left
        else:
            current = current.right
    return None
\end{lstlisting}
\end{algo}

\subsubsection{Insert}
\begin{algo}
\begin{lstlisting}
def insert(self, key):
    def _insert(node, key):
        if node is None:
            return TreeNode(key)
        if key < node.val:
            node.left = _insert(node.left, key)
        elif key > node.val:
            node.right = _insert(node.right, key)
        return node
    self.root = _insert(self.root, key)
\end{lstlisting}
\end{algo}

\subsubsection{Delete}
\begin{algo}
\begin{lstlisting}
def delete(self, key):
    def _delete(node, key):
        if node is None:
            return None
        if key < node.val:
            node.left = _delete(node.left, key)
        elif key > node.val:
            node.right = _delete(node.right, key)
        else:
            # Node with one child or no child
            if node.left is None:
                return node.right
            elif node.right is None:
                return node.left
            # Node with two children
            temp = self._find_min(node.right)
            node.val = temp.val
            node.right = _delete(node.right, temp.val)
        return node
    self.root = _delete(self.root, key)
\end{lstlisting}
\end{algo}
    
\subsubsection{Find Min}
\begin{algo}
\begin{lstlisting}
def find_min(self, node):
    while node.left is not None:
        node = node.left
    return node
\end{lstlisting}
\end{algo}

\subsubsection{Find Max}
\begin{algo}
\begin{lstlisting}
def find_max(self, node):
    while node.right is not None:
        node = node.right
    return node
\end{lstlisting}
\end{algo}

\subsubsection{DFS In-order Traversal (Left → Root → Right)}
\begin{definition}
    Visit the left subtree, then the root, and finally the right subtree.
    \begin{itemize}
        \item Used for retrieving elements in sorted order from a BST.
    \end{itemize}
\end{definition}

\begin{algo}
\begin{lstlisting} 
def inorder(node):
    if node:
        inorder(node.left)
        print(node.val)
        inorder(node.right)
\end{lstlisting}  
\end{algo}
\newpage

\subsubsection{DFS Pre-order Traversal (Root → Left → Right)}
\begin{definition}
    Visit the root first, then the left subtree, and finally the right subtree.
    \begin{itemize}
        \item Useful for copying or serializing the tree.
    \end{itemize}
\end{definition}

\begin{algo} 
\begin{lstlisting}
def preorder(node):
    if node:
        print(node.val)
        preorder(node.left)
        preorder(node.right)
\end{lstlisting}
\end{algo}

\subsubsection{DFS Post-order Traversal (Left → Right → Root)}
\begin{definition}
    Visit the left subtree, then the right subtree, and finally the root.
    \begin{itemize}
        \item Useful for deleting or freeing nodes in memory.
    \end{itemize}
\end{definition}

\begin{algo}
\begin{lstlisting}
def postorder(node):
    if node:
        postorder(node.left)
        postorder(node.right)
        print(node.val)
\end{lstlisting}
\end{algo}

\subsubsection{BFS Level-order Traversal (Top → Bottom, Left → Right)}
\begin{definition}
    Visit nodes level-level from top to bottom \& left to right.
    \begin{itemize}
        \item Useful for finding shortest paths or visualizing layers of a tree.
    \end{itemize}
\end{definition}

\begin{algo}
\begin{lstlisting}
from collections import deque

def level_order(root):
    if not root:
        return

    queue = deque([root])
    while queue:
        node = queue.popleft()
        print(node.val)

        if node.left:
            queue.append(node.left)
        if node.right:
            queue.append(node.right)
\end{lstlisting}
\end{algo}
\newpage

\subsubsection{Common Problems}
\begin{summary}
    \begin{center}
        \begin{tabular}{ll}
            \toprule
            \textbf{Problem} & \textbf{Description:} \\
            **226. Invert Binary Tree & Given a binary tree, invert it. \\
            \multicolumn{2}{p{\linewidth}}{
                \begin{itemize}
                    \item \textbf{Base Case:} If \texttt{root} is \texttt{None}, return \texttt{None}.
                
                    \item \textbf{Swap Subtrees:} Swap the left and right children of the current \texttt{root}.
                
                    \item \textbf{Recursive Inversion:}
                    \begin{itemize}
                        \item Recursively invert the left subtree by calling \texttt{invertTree(root.left)}.
                        \item Recursively invert the right subtree by calling \texttt{invertTree(root.right)}.
                    \end{itemize}
                
                    \item \textbf{Return Result:} Return the current \texttt{root} after its subtrees have been inverted.
                \end{itemize}                
            } \\
            \midrule
            **104. Maximum Depth of Binary Tree & Given a binary tree, find its maximum depth. \\
            \multicolumn{2}{p{\linewidth}}{
                \begin{itemize}
                    \item \textbf{Recursive DFS:}
                    \begin{itemize}
                        \item \textbf{Base Case:} If \texttt{root} is \texttt{None}, return \texttt{0}.
                    
                        \item \textbf{Recursive Depth Calculation:}
                        \begin{itemize}
                            \item Recursively compute the maximum depth of the left subtree by calling \texttt{maxDepth(root.left)}.
                            \item Recursively compute the maximum depth of the right subtree by calling \texttt{maxDepth(root.right)}.
                        \end{itemize}
                    
                        \item \textbf{Return Result:} Return $1$ plus the maximum of the left and right subtree depths.
                    \end{itemize}
                \end{itemize}
            } \\
            \midrule
            **543. Diameter of Binary Tree & Given a binary tree, find its diameter. \\
            \multicolumn{2}{p{\linewidth}}{
                \begin{itemize}
                    \item \textbf{Initialization:} Initialize a variable \texttt{res = 0} to store the maximum diameter found.
                
                    \item \textbf{Depth-First Search (DFS):} 
                    \begin{itemize}
                        \item If \texttt{root} is \texttt{None}, return $0$.
                        \item Recursively compute the left subtree depth by calling \texttt{dfs(root.left)}.
                        \item Recursively compute the right subtree depth by calling \texttt{dfs(root.right)}.
                        \item Update \texttt{res} as the maximum of its current value and \texttt{left + right}.
                        \item Return $1 + \max(\texttt{left}, \texttt{right})$ to represent the height of the current subtree.
                    \end{itemize}
                
                    \item \textbf{Result:}
                    \begin{itemize}
                        \item Call \texttt{dfs(root)} to start the recursion from the root node.
                        \item Return the final value of \texttt{res}, which represents the diameter of the tree.
                    \end{itemize}
                \end{itemize}                
            } \\
            \midrule
            110. Balanced Binary Tree & Given a binary tree, check if it is height-balanced. \\
            \multicolumn{2}{p{\linewidth}}{
                \begin{itemize}
                    \item \textbf{Recursive Depth-First Search (DFS):}
                    \begin{itemize}
                        \item If \texttt{root} is \texttt{None}, return \texttt{[True, 0]} (tree is balanced with height $0$).
                        \item Recursively check the left and right subtrees by calling \texttt{dfs(root.left)} and \texttt{dfs(root.right)}.
                        \item A node is balanced if:
                        \begin{itemize}
                            \item Both left and right subtrees are balanced.
                            \item The height difference between the left and right subtrees is at most $1$.
                        \end{itemize}
                        \item Return a list \texttt{[isBalanced, height]}, where:
                        \begin{itemize}
                            \item \texttt{isBalanced} is a boolean indicating subtree balance.
                            \item \texttt{height} is $1 + \max(\texttt{left height}, \texttt{right height})$.
                        \end{itemize}
                    \end{itemize}
                
                    \item \textbf{Return Result:} Return 1st element of the result from \texttt{dfs(root)}, indicating whether entire tree is bal.
                \end{itemize}
            } \\
            \bottomrule
        \end{tabular}
    \end{center}
\end{summary}
\newpage

\begin{summary}
    \begin{center}
        \begin{tabular}{ll}
            \toprule
            \textbf{Problem} & \textbf{Description:} \\
            100. Same Tree & Given two binary trees, check if they are the same. \\
            \multicolumn{2}{p{\linewidth}}{
                \begin{itemize}
                    \item \textbf{Base Cases:}
                    \begin{itemize}
                        \item If both \texttt{p} and \texttt{q} are \texttt{None}, return \texttt{True}.
                        \item If only one of \texttt{p} or \texttt{q} is \texttt{None}, or their values differ, return \texttt{False}.
                    \end{itemize}
                
                    \item \textbf{Recursive Comparison:}
                    \begin{itemize}
                        \item Recursively check if the left subtrees \texttt{p.left} and \texttt{q.left} are identical.
                        \item Recursively check if the right subtrees \texttt{p.right} and \texttt{q.right} are identical.
                        \item Return \texttt{True} only if both left and right subtree comparisons return \texttt{True}.
                    \end{itemize}
                \end{itemize}                
            } \\
            \midrule
            235. Lowest Common Ancestor of a BST & Given a BST and two nodes, find their lowest common ancestor. \\
            \multicolumn{2}{p{\linewidth}}{
                \begin{itemize}
                    \item \textbf{Initialization:} Set \texttt{cur} to the \texttt{root} node of the tree.
                
                    \item \textbf{Iterative Traversal:} While \texttt{cur} is not \texttt{None}:
                        \begin{itemize}
                            \item If both \texttt{p.val} and \texttt{q.val} are greater than \texttt{cur.val}, move to \texttt{cur.right}.
                            \item Else if both \texttt{p.val} and \texttt{q.val} are less than \texttt{cur.val}, move to \texttt{cur.left}.
                            \item Otherwise, \texttt{cur} is the split point where paths to \texttt{p} and \texttt{q} diverge, and thus \texttt{cur} is the lowest common ancestor (LCA).
                        \end{itemize}
                
                    \item \textbf{Return Result:} Return the node \texttt{cur} when the split point is found.
                \end{itemize}                
            } \\
            \midrule
            102. Binary Tree Level Order Traversal & Given a binary tree, return its level order traversal. \\
            \multicolumn{2}{p{\linewidth}}{
                \begin{itemize}
                    \item \textbf{Initialization:} Create an empty list \texttt{res} to store nodes level-by-level.
                
                    \item \textbf{Depth-First Search (DFS):} Define a recursive function \texttt{dfs(node, depth)}:
                        \begin{itemize}
                            \item If \texttt{node} is \texttt{None}, return immediately.
                            \item If \texttt{depth} equals the length of \texttt{res}, append a new empty list for this depth level.
                            \item Append \texttt{node.val} to the corresponding depth list.
                            \item Recursively call \texttt{dfs(node.left, depth + 1)} and \texttt{dfs(node.right, depth + 1)}.
                        \end{itemize}
                
                    \item \textbf{Return Result:}
                    \begin{itemize}
                        \item Call \texttt{dfs(root, 0)} to start traversal.
                        \item Return the list \texttt{res} containing all levels.
                    \end{itemize}
                \end{itemize}                
            } \\
            \midrule 
            98. Validate Binary Search Tree & Given a binary tree, check if it is a valid BST. \\
            \multicolumn{2}{p{\linewidth}}{
                \begin{itemize}
                    \item \textbf{Initialization:}
                    \begin{itemize}
                        \item If \texttt{root} is \texttt{None}, return \texttt{True}.
                        \item Initialize a queue \texttt{q} with a tuple containing the root node and its valid range \texttt{(-$\infty$, $\infty$)}.
                    \end{itemize}
                
                    \item \textbf{Breadth-First Search (BFS) Traversal:} While the queue is not empty:
                        \begin{itemize}
                            \item Dequeue a node along with its valid value bounds \texttt{(left, right)}.
                            \item If the node’s value is not strictly between \texttt{left} and \texttt{right}, return \texttt{False}.
                            \item If the node has a left child, enqueue it with updated bounds \texttt{(left, node.val)}.
                            \item If the node has a right child, enqueue it with updated bounds \texttt{(node.val, right)}.
                        \end{itemize}
                
                    \item \textbf{Return Result:} After completing traversal without violations, return \texttt{True}.
                \end{itemize}                              
            } \\
            \midrule
            230. Kth Smallest Element in a BST & Given a BST and an integer k, find the kth smallest element. \\
            \multicolumn{2}{p{\linewidth}}{
                \begin{itemize}
                    \item \textbf{Initialization:} Create an empty list \texttt{arr} to store node values in ascending order.
                
                    \item \textbf{Depth-First Search (DFS) In-Order Traversal:} Define a recursive function \texttt{dfs(node)}:
                        \begin{itemize}
                            \item If \texttt{node} is \texttt{None}, return immediately.
                            \item Recursively call \texttt{dfs(node.left)} to visit the left subtree.
                            \item Append \texttt{node.val} to \texttt{arr}.
                            \item Recursively call \texttt{dfs(node.right)} to visit the right subtree.
                        \end{itemize}
                
                    \item \textbf{Return Result:} DFS starting from the \texttt{root}, return \texttt{arr[k-1]}, which is the $k^{\text{th}}$ smallest element.
                \end{itemize}   
            } \\
            \bottomrule
        \end{tabular}
    \end{center}
\end{summary}
\newpage

\subsubsection{BST-based Sets and Maps}

\begin{summary}
    \begin{itemize}
        \item \textbf{BST Set}: Stores unique values in sorted order. Supports insert, search, delete.
        \item \textbf{BST Map}: Associates keys with values, maintaining keys in sorted order.
        \item Can be implemented using self-balancing trees (e.g., AVL, Red-Black Tree) for O(log n) operations.
        \item Useful for range queries, floor/ceiling lookups, and ordered iteration.
    \end{itemize}
\end{summary}

\begin{algo}
\begin{lstlisting}
class BSTSet:
    def __init__(self):
        self.root = None

    def add(self, val):
        self.root = insert_bst(self.root, val)

    def contains(self, val):
        return search_bst(self.root, val) is not None

    def remove(self, val):
        self.root = delete_bst(self.root, val)

class BSTMap:
    def __init__(self):
        self.root = None

    def put(self, key, value):
        self.root = self._put(self.root, key, value)

    def _put(self, node, key, value):
        if not node:
            return TreeNode((key, value))
        if key < node.val[0]:
            node.left = self._put(node.left, key, value)
        elif key > node.val[0]:
            node.right = self._put(node.right, key, value)
        else:
            node.val = (key, value)
        return node

    def get(self, key):
        node = self.root
        while node:
            if key < node.val[0]:
                node = node.left
            elif key > node.val[0]:
                node = node.right
            else:
                return node.val[1]
        return None
\end{lstlisting}
\end{algo}