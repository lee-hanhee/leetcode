\subsection{2D Convolution Operations}
\begin{notes}
    \begin{enumerate}
        \item \textbf{1. Output Dimensions}

        The output height and width of a 2D convolution are given by:
    
        \[
        \text{out\_height} = \left\lfloor \frac{\text{in\_height} + 2 \cdot \text{padding}_h - \text{effective\_kernel}_h}{\text{stride}_h} \right\rfloor + 1
        \]
    
        \[
        \text{out\_width} = \left\lfloor \frac{\text{in\_width} + 2 \cdot \text{padding}_w - \text{effective\_kernel}_w}{\text{stride}_w} \right\rfloor + 1
        \]
    
        \item \textbf{2. Effective Kernel Size (with Dilation)}
    
        The effective kernel size when dilation is applied:
    
        \[
        \text{effective\_kernel}_h = \text{kernel\_height} + (\text{kernel\_height} - 1) \cdot (\text{dilation}_h - 1)
        \]
    
        \[
        \text{effective\_kernel}_w = \text{kernel\_width} + (\text{kernel\_width} - 1) \cdot (\text{dilation}_w - 1)
        \]
    
        \item \textbf{3. Convolution Operation (Batch, Channel-aware)}
    
        The general convolution operation for a batch of input tensors is:
    
        \[
        \text{output}[b, c_{\text{out}}, h_{\text{out}}, w_{\text{out}}] = \sum_{c_{\text{in}}} \sum_{k_h} \sum_{k_w}
        \left(
        \text{input}[b, c_{\text{in}}, h_{\text{in}} + k_h \cdot \text{dilation}_h, w_{\text{in}} + k_w \cdot \text{dilation}_w]
        \cdot
        \text{filter}[c_{\text{out}}, c_{\text{in}}, k_h, k_w]
        \right)
        \]
    
        where:
    
        \[
        h_{\text{in}} = h_{\text{out}} \cdot \text{stride}_h, \quad w_{\text{in}} = w_{\text{out}} \cdot \text{stride}_w
        \]
    \end{enumerate}
\end{notes}

\subsection{Common Problems}
\begin{summary}
    \begin{center}
        \begin{tabular}{ll}
            \toprule
            \textbf{Problem} & \textbf{Description} \\
            \midrule
            661. Image Smoother & Given an image represented by a 2D array, \\
            & smooth the image by averaging the pixel values \\
            & of each pixel and its neighbors. \\
            \multicolumn{2}{p{\linewidth}}{
                \begin{itemize}
                    \item Loop through the cols and rows of the image, then 
                    \begin{itemize}
                        \item $\text{total sum for each pixel} = \sum_{x,y \in \text{neighbours}} \text{image}[x][y] = \sum_{x=i-1}^{i+1} \sum_{y=j-1}^{j+1} \text{image}[x][y]$
                        \begin{itemize}
                            \item If $x$ or $y$ is out of bounds, ignore it.
                        \end{itemize}
                        \item $\text{count} = \sum_{x=i-1}^{i+1} \sum_{y=j-1}^{j+1} 1$
                        \item $\text{average} = \text{total sum} // \text{count}$
                    \end{itemize}
                    \item $\text{result}[i][j] = \text{average}$
                \end{itemize}
            } \\
            \midrule
            832. Flipping an Image & Given a binary matrix, flip the image \\
            & horizontally and invert it. \\
            \multicolumn{2}{p{\linewidth}}{
                \begin{itemize}
                    \item Loop through the rows of the image, then use .reverse() to flip the row horizontally.
                    \item Double for loop to invert image (change 0 to 1 and 1 to 0).
                \end{itemize}
            } \\
            \midrule
            48. Rotate Image & Given an n x n 2D matrix, rotate the image \\
            & 90 degrees clockwise. \\
            \multicolumn{2}{p{\linewidth}}{
                \begin{itemize}
                    \item Transpose the matrix (swap rows and columns) if $i<j$, then $\text{matrix}[i][j] \overset{\text{swap}}{\iff} \text{matrix}[j][i]$.
                    \item Reverse each row.
                \end{itemize}
            } \\
            \midrule
            **835. Image Overlap & Given two images represented by 2D arrays, \\
            & find the maximum overlap between the two images. \\
            \multicolumn{2}{p{\linewidth}}{
                \begin{itemize}
                    \item Try all possible translations of img1. 
                    \item For each translation, calculate the overlap with img2.
                \end{itemize}
            }
        \end{tabular}
    \end{center}
\end{summary}