\subsection{When to Use?}
\begin{summary}
    \begin{itemize}
        \item If we need to find a pair of elements that satisfy a condition.
        \item If we need to find a subarray that satisfies a condition.
    \end{itemize}
\end{summary}

\subsection{Slow and Fast Pointers}
\begin{algo}
    \begin{enumerate}
        \item 
    \end{enumerate}
\end{algo}

\subsubsection{Common Problems}
\begin{summary}
    \begin{center}
        \begin{tabular}{ll}
            \toprule
            \textbf{Problem} & \textbf{Description:} \\
            \midrule
            15. 3Sum & Given an array of integers, return all the triplets \\
            & [nums[i], nums[j], nums[k]] s.t. i != j, i != k, and j != k. \\
            \multicolumn{2}{p{\linewidth}}{
                \begin{itemize}
                    \item \textbf{Tricks:}
                \end{itemize}
            } \\
            \midrule
            125. Valid Palindrome & Given a string, determine if it is a palindrome, \\
            & considering only alphanumeric characters and ignoring cases. \\
            \multicolumn{2}{p{\linewidth}}{
                \begin{itemize}
                    \item \texttt{s\_new = ''.join(char.lower() for char in s if char.isalnum())} to remove non-alphanumeric and lowercase. 
                    \item Use front and back pointers. If they not equal, return False. If equal move both pointers.
                \end{itemize}
            } \\
            \midrule
            167. Two Sum II - Input array is sorted & Given an array of integers that is already sorted in ascending order, \\
            & find two numbers such that they add up to a target. \\
            \multicolumn{2}{p{\linewidth}}{
                \begin{itemize}
                    \item Use front and back pointers. If > target, move back pointer left. If < target, move front pointer right.
                \end{itemize}
            } \\
            \midrule
            \bottomrule
        \end{tabular}
    \end{center}
\end{summary}
\newpage

\subsection{Front and Back Pointers}
\begin{algo}
    \begin{enumerate}
        \item Initialize two pointers, one at the front and one at the back of the array.
    \end{enumerate}
\end{algo}

\subsubsection{Common Problems}
\begin{summary}
    \begin{center}
        \begin{tabular}{ll}
            \toprule
            \textbf{Problem} & \textbf{Description:} \\
            \midrule
            15. 3Sum & Given an array of integers, return all the triplets \\
            & [nums[i], nums[j], nums[k]] s.t. i != j, i != k, and j != k. \\
            \multicolumn{2}{p{\linewidth}}{
                \begin{itemize}
                    \item \textbf{Tricks:}
                \end{itemize}
            } \\
            \midrule
            125. Valid Palindrome & Given a string, determine if it is a palindrome, \\
            & considering only alphanumeric characters and ignoring cases. \\
            \multicolumn{2}{p{\linewidth}}{
                \begin{itemize}
                    \item \texttt{s\_new = ''.join(char.lower() for char in s if char.isalnum())} to remove non-alphanumeric and lowercase. 
                    \item Use front and back pointers. If they not equal, return False. If equal move both pointers.
                \end{itemize}
            } \\
            \midrule
            167. Two Sum II - Input array is sorted & Given an array of integers that is already sorted in ascending order, \\
            & find two numbers such that they add up to a target. \\
            \multicolumn{2}{p{\linewidth}}{
                \begin{itemize}
                    \item Use front and back pointers. If > target, move back pointer left. If < target, move front pointer right.
                \end{itemize}
            } \\
            \midrule
            \bottomrule
        \end{tabular}
    \end{center}
\end{summary}
