\begin{summary}
    \begin{center}
        \begin{tabular}{lll}
            \toprule
            \textbf{Algorithm} & \textbf{Time Complexity} & \textbf{Space Complexity} \\
            \midrule
            BFS & $O(V + E)$ & $O(V)$ \\
            \midrule
            DFS & $O(V + E)$ & $O(V)$ \\
            \midrule
            Topological Sort (DFS) & $O(V + E)$ & $O(V)$ \\
        \end{tabular}
    \end{center}
\end{summary}

\subsection{Breadth-First Search (BFS)}
\begin{summary}
    \begin{itemize}
        \item Use when exploring nodes layer-by-layer, typically in unweighted graphs or grids.
        \item Ideal for finding the shortest path, level order traversal, or minimum number of steps.
        \item Queue-based traversal ensures nodes are visited in order of increasing distance from the source.
    \end{itemize}
\end{summary}

\begin{algo}
\begin{lstlisting}
from collections import deque

def bfs(start, graph):
    visited = set()
    queue = deque([start])
    visited.add(start)

    while queue:
        node = queue.popleft() # FIFO (BFS)

        for neighbor in graph[node]:
            if neighbor not in visited:
                visited.add(neighbor)
                queue.append(neighbor)
\end{lstlisting}
\end{algo}
\newpage

\subsubsection{Common Problems}
\begin{summary}
\end{summary}
\newpage

\subsection{Depth-First Search (DFS)}

\begin{summary}
    \begin{itemize}
        \item Use when traversing all nodes or paths in \textbf{trees}, \textbf{graphs}, or \textbf{matrices}.
        \item Ideal for problems involving \textbf{backtracking}, \textbf{recursion}, or exploring all \textbf{connected components}.
        \item Can be implemented recursively or iteratively with a stack.
        \item Maintain a \texttt{visited} set or matrix to avoid revisiting nodes.
        \item Useful for \textbf{topological sorting}, \textbf{cycle detection}, and \textbf{pathfinding}.
    \end{itemize}
\end{summary}

\begin{algo}
\begin{lstlisting}
from collections import deque

def dfs(start, graph):
    visited = set()
    stack = deque([start])
    visited.add(start)

    while stack:
        node = stack.pop()  # LIFO (DFS)

        for neighbor in graph[node]:
            if neighbor not in visited:
                visited.add(neighbor)
                stack.append(neighbor) 
\end{lstlisting}
    \end{algo}

\begin{algo}
\begin{lstlisting}
def dfs(node, visited):
    if node in visited:
        return

    visited.add(node)

    for neighbor in graph[node]:
        dfs(neighbor, visited)
\end{lstlisting}
\end{algo}

\newpage

\subsubsection{Common Problems}
\begin{summary}
    \begin{center}
        \begin{tabular}{ll}
            \toprule
            \textbf{Problem} & \textbf{Description:} \\
            \midrule
            200. Number of Islands & Given a 2D grid of '1's (land) and '0's (water), count the number of islands. \\
            \multicolumn{2}{p{\linewidth}}{
                \begin{itemize}
                    \item Use DFS or BFS to explore all connected '1's and mark them as visited.
                    \item Increment the island count for each unvisited '1'.
                \end{itemize}
            } \\
            \midrule
            695. Max Area of Island & Given a 2D grid of '1's (land) and '0's (water), find the maximum area of an island. \\
            \multicolumn{2}{p{\linewidth}}{
                \begin{itemize}
                    \item Use DFS or BFS to explore all connected '1's and calculate the area.
                    \item Keep track of the maximum area encountered during the traversal.
                \end{itemize}
            } \\
            \midrule
            79. Word Search & Given a 2D board and a word, check if the word exists in the grid. \\
            \multicolumn{2}{p{\linewidth}}{
                \begin{itemize}
                    \item Use DFS to explore all possible paths in the grid.
                    \item Mark cells as visited to avoid revisiting.
                    \item Backtrack if the current path does not lead to a solution.
                \end{itemize}
            } \\
        \end{tabular}
    \end{center}
\end{summary}
\newpage

\subsection{Topological Sort}
\begin{summary}
    \begin{itemize}
        \item \textbf{Overview:} Produces a total ordering from partial ordering.
        \item \textbf{DAG:} $G=(V,E)$ must be a DAG to produce a valid topological sorting. 
        \item Given a DAG, create a linear (total) order out of the partial order $\rightarrow$ "serialize" these events 
        \begin{itemize}
            \item \textbf{Intuition:} Arranges the vertices of a DAG in a linear order such that for every directed edge \( u \to v \), vertex \( u \) appears before \( v \).
        \end{itemize}
    \end{itemize}
\end{summary}

\begin{algo}
\begin{lstlisting}
from collections import defaultdict, deque

def topological_sort(num_nodes, edges):
    # Build adjacency list and in-degree count
    graph = defaultdict(list)
    in_degree = [0] * num_nodes

    for u, v in edges:
        graph[u].append(v)
        in_degree[v] += 1

    # Start with all nodes that have in-degree 0
    queue = deque([i for i in range(num_nodes) if in_degree[i] == 0])
    topo_order = []

    while queue:
        node = queue.popleft()
        topo_order.append(node)

        for neighbor in graph[node]:
            in_degree[neighbor] -= 1
            if in_degree[neighbor] == 0:
                queue.append(neighbor)

    # If not all nodes are processed, there is a cycle
    if len(topo_order) != num_nodes:
        return []  # or raise an error

    return topo_order
\end{lstlisting}
        
\end{algo}