\subsection{collections.deque}
\begin{summary}
\begin{lstlisting}
from collections import deque
\end{lstlisting}

\begin{lstlisting}
deque(iterable=None, maxlen=None)
\end{lstlisting}
\begin{itemize}
    \item Creates a new deque object initialized with elements from \texttt{iterable}.
    \item \textbf{Arguments:} \texttt{iterable}: optional iterable of elements; \texttt{maxlen}: maximum number of elements.
\end{itemize}

\begin{lstlisting}
d.append(x)
\end{lstlisting}
\begin{itemize}
    \item Adds \texttt{x} to the right end of the deque.
    \item \textbf{Arguments:} \texttt{x}: element to append.
\end{itemize}

\begin{lstlisting}
d.appendleft(x)
\end{lstlisting}
\begin{itemize}
    \item Adds \texttt{x} to the left end of the deque.
    \item \textbf{Arguments:} \texttt{x}: element to append.
\end{itemize}

\begin{lstlisting}
d.pop()
\end{lstlisting}
\begin{itemize}
    \item Removes and returns the rightmost element.
    \item \textbf{Raises:} \texttt{IndexError} if deque is empty.
\end{itemize}

\begin{lstlisting}
d.popleft()
\end{lstlisting}
\begin{itemize}
    \item Removes and returns the leftmost element.
    \item \textbf{Raises:} \texttt{IndexError} if deque is empty.
\end{itemize}

\begin{lstlisting}
d.extend(iterable)
\end{lstlisting}
\begin{itemize}
    \item Appends elements from \texttt{iterable} to the right side.
    \item \textbf{Arguments:} \texttt{iterable}: iterable of elements to append.
\end{itemize}

\begin{lstlisting}
d.extendleft(iterable)
\end{lstlisting}
\begin{itemize}
    \item Appends elements from \texttt{iterable} to the left side (in reverse order).
    \item \textbf{Arguments:} \texttt{iterable}: iterable of elements to append.
\end{itemize}

\begin{lstlisting}
d.rotate(n=1)
\end{lstlisting}
\begin{itemize}
    \item Rotates the deque \texttt{n} steps to the right (left if negative).
    \item \textbf{Arguments:} \texttt{n}: number of steps to rotate.
\end{itemize}

\begin{lstlisting}
d.clear()
\end{lstlisting}
\begin{itemize}
    \item Removes all elements from the deque.
    \item \textbf{Postcondition:} deque is empty.
\end{itemize}

\begin{lstlisting}
d.count(x)
\end{lstlisting}
\begin{itemize}
    \item Counts occurrences of \texttt{x} in the deque.
    \item \textbf{Arguments:} \texttt{x}: element to count.
\end{itemize}

\begin{lstlisting}
d.remove(value)
\end{lstlisting}
\begin{itemize}
    \item Removes the first occurrence of \texttt{value}.
    \item \textbf{Raises:} \texttt{ValueError} if value is not present.
\end{itemize}

\end{summary}
\newpage

\subsection{Stack}
\begin{summary}
    Data structure that follows Last-In-First-Out (LIFO) order for inserting and removing elements.
    \begin{itemize}
        \item \textbf{Array or Linked List:} Used to maintain the linear order of elements.
        \item \textbf{Top Pointer:} Points to the most recently inserted element.
    \end{itemize}    
\end{summary}

\subsubsection{When to Use?}
\begin{summary}
    \begin{itemize}
        \item Function call management using a call stack.
        \item Reversing sequences or backtracking algorithms.
        \item Syntax parsing and expression evaluation.
    \end{itemize}
\end{summary}

\subsubsection{Operations}
\begin{summary}
    \begin{center}
        \begin{tabular}{lc}
            \toprule
            \textbf{Operation} & \textbf{Time Complexity (WC)} \\
            \midrule
            Push   & $O(1)$ \\
            Pop    & $O(1)$ \\
            Peek   & $O(1)$ \\
            IsEmpty & $O(1)$ \\
            \bottomrule
        \end{tabular}
    \end{center}
\end{summary}

\begin{algo}
\begin{lstlisting}[language=Python]
class Stack:
    def __init__(self):
        self.items = []  # Internal array to store elements

    def push(self, item):
        self.items.append(item)  # Add item to top

    def pop(self):
        if not self.is_empty():
            return self.items.pop()  # Remove and return top element

    def peek(self):
        if not self.is_empty():
            return self.items[-1]  # Return top element without removing

    def is_empty(self):
        return len(self.items) == 0
\end{lstlisting}
\end{algo}
\newpage

\subsubsection{Common Problems}
\begin{summary}
    \begin{center}
        \begin{tabular}{ll}
            \toprule
            \textbf{Problem} & \textbf{Description} \\
            \midrule
            20. Valid Parentheses & Check if parentheses are balanced. \\
            \multicolumn{2}{p{\linewidth}}{
                \begin{itemize}
                    \item \textbf{Initialization:}
                    \begin{itemize}
                        \item Create an empty \texttt{stack} to track unmatched opening brackets.
                        \item Define a \texttt{closeToOpen} mapping from closing brackets to corresponding opening brackets.
                    \end{itemize}
                
                    \item \textbf{String Traversal:}
                    \begin{itemize}
                        \item For each character:
                        \begin{itemize}
                            \item If it is a closing bracket:
                            \begin{itemize}
                                \item Check if the stack is non-empty and the top of the stack matches the corresponding opening bracket. If yes, pop the stack; otherwise, return \texttt{False}.
                            \end{itemize}
                            \item If it is an opening bracket, push it onto the stack.
                        \end{itemize}
                    \end{itemize}
                
                    \item \textbf{Final Check:} Return \texttt{True} if the stack is empty (all brackets matched), else \texttt{False}.
                
                    \item \textbf{Key Insight:}  The stack ensures that brackets are matched in the correct type and order, guaranteeing validity through last-in, first-out (LIFO) behavior.
                \end{itemize}                
            } \\
            \midrule
            150. Evaluate Reverse Polish Notation & Evaluate expression in postfix notation. \\
            \multicolumn{2}{p{\linewidth}}{
                \begin{itemize}
                    \item \textbf{Initialization:} Create an empty \texttt{stack} to store intermediate operands and results.
                
                    \item \textbf{Token Traversal:}
                    \begin{itemize}
                        \item For each token in the input list:
                        \begin{itemize}
                            \item If the token is an operator (\texttt{+}, \texttt{-}, \texttt{*}, \texttt{/}):
                            \begin{itemize}
                                \item Pop the top two elements from the stack.
                                \item Apply the operator in the correct order (note: for subtraction and division, order matters).
                                \item Push the result back onto the stack.
                            \end{itemize}
                            \item If the token is a number:
                            \begin{itemize}
                                \item Convert it to an integer and push it onto the stack.
                            \end{itemize}
                        \end{itemize}
                    \end{itemize}
                
                    \item \textbf{Final Result:} Return the top element of the stack, which contains the final evaluated value.
                
                    \item \textbf{Key Insight:} Reverse Polish Notation (postfix notation) allows expressions to be evaluated using a stack without parentheses by always applying operations to the two most recent operands.
                \end{itemize}                              
            } \\
            \midrule
            22. Generate Parentheses & Generate all combinations of valid parentheses. \\
            \multicolumn{2}{p{\linewidth}}{
                \begin{itemize}
                    \item \textbf{Initialization:} Create \texttt{stack} to build the current sequence and \texttt{res} list to store valid sequences.
                
                    \item \textbf{Recursive Backtracking:} Define a recursive function \texttt{backtrack(openN, closedN)}:
                    \begin{itemize}
                        \item If both \texttt{openN} and \texttt{closedN} equal \texttt{n}, a complete valid sequence is formed. Append it to \texttt{res}.
                        \item If \texttt{openN < n}, add an opening bracket \texttt{"("}, recurse, then backtrack by removing it.
                        \item If \texttt{closedN < openN}, add a closing bracket \texttt{")"}, recurse, then backtrack by removing it.
                    \end{itemize}
                
                    \item \textbf{Initial Call:} Start the recursion with \texttt{openN = 0}, \texttt{closedN = 0}.
                
                    \item \textbf{Return Result:} Return the list \texttt{res} containing all valid combinations.
                
                    \item \textbf{Key Insight:} Maintain the constraint that at any point, the number of closing brackets must not exceed the number of opening brackets to ensure sequence validity.
                \end{itemize}
            } \\
            \bottomrule
        \end{tabular}
    \end{center}
\end{summary}
\newpage

\begin{summary}
    \begin{center}
        \begin{tabular}{ll}
            \toprule
            \textbf{Problem} & \textbf{Description} \\
            \midrule     
            739. Daily Temperatures & Find days until a warmer temperature. \\
            \multicolumn{2}{p{\linewidth}}{
                \begin{itemize}
                    \item \textbf{Initialization:}
                    \begin{itemize}
                        \item Create a result list \texttt{res} initialized with zeros, having the same length as \texttt{temperatures}.
                        \item Create an empty \texttt{stack} to store pairs of (temperature, index).
                    \end{itemize}
                
                    \item \textbf{Array Traversal:}
                    \begin{itemize}
                        \item Iterate through each temperature and its index:
                        \begin{itemize}
                            \item While the stack is not empty and the current temperature is greater than the temperature at the top of the stack:
                            \begin{itemize}
                                \item Pop the stack, and for the popped index, set the result as the difference between the current index and the popped index.
                            \end{itemize}
                            \item Push the current temperature and index onto the stack.
                        \end{itemize}
                    \end{itemize}
                
                    \item \textbf{Return Result:} Return the result list \texttt{res}, which contains the number of days to wait for a warmer temperature for each day.
                
                    \item \textbf{Key Insight:} A monotonic decreasing stack is used to efficiently find, for each day, the next day with a higher temperature, achieving linear $O(n)$ time complexity.
                \end{itemize}                           
            } \\
            \midrule               
            \bottomrule
        \end{tabular}
    \end{center}
\end{summary}
\newpage

\subsection{Queue}
\begin{summary}
    Data structure that follows First-In-First-Out (FIFO) order for inserting and removing elements.
    \begin{itemize}
        \item \textbf{Array or Linked List:} Used to store elements in sequence.
        \item \textbf{Front and Rear Pointers:} Track the ends for dequeue and enqueue operations.
    \end{itemize}    
\end{summary}

\subsubsection{When to Use?}
\begin{summary}
    \begin{itemize}
        \item Scheduling processes in operating systems.
        \item Handling asynchronous data (e.g., IO Buffers, Event Queues).
        \item Breadth-First Search in graphs or trees.
    \end{itemize}
\end{summary}

\subsubsection{Operations}
\begin{summary}
    \begin{center}
        \begin{tabular}{lc}
            \toprule
            \textbf{Operation} & \textbf{Time Complexity (WC)} \\
            \midrule
            Enqueue  & $O(1)$ \\
            Dequeue  & $O(n)$ \\
            Peek     & $O(1)$ \\
            IsEmpty  & $O(1)$ \\
            \bottomrule
        \end{tabular}
    \end{center}
\end{summary}

\begin{algo}
\begin{lstlisting}
class Queue:
    def __init__(self):
        self.items = []  # Internal array to store elements

    def enqueue(self, item):
        self.items.append(item)  # Add item to the rear

    def dequeue(self):
        if not self.is_empty():
            return self.items.pop(0)  # Remove and return the front element

    def peek(self):
        if not self.is_empty():
            return self.items[0]  # Return front element without removing

    def is_empty(self):
        return len(self.items) == 0
\end{lstlisting}
\end{algo}